\documentclass[12pt,fleqn]{article}

% Add commit information from Git to the pdf. It is automatically
% generated with the R script 'tex/git and can be run from
% the command line in the project's top directory via:
%
% $ tex/git
%
% If VERSION.tex does not exist we can't add information from
% Git, so we'll use today's date as a fallback.

\IfFileExists{./VERSION.tex}{\input{VERSION}}{%
\providecommand\VERSION{\today}}

\usepackage{
  amsfonts,
  amsmath,
  amssymb,
  amsthm,
  booktabs,
  fancyhdr,
  graphicx,
  newtxmath,
  newtxtext,
}

\frenchspacing
\usepackage[margin=1in]{geometry}

% Add version information to the paper's date and its page heading.
\date{\VERSION}
\pagestyle{fancy}
\renewcommand{\sectionmark}[1]{\markboth{}{\footnotesize{\thesection. #1}}}
\renewcommand{\subsectionmark}[1]{\markboth{}{\footnotesize{\thesubsection. #1}}}
\renewcommand{\headrulewidth}{0pt}
\renewcommand{\footrulewidth}{0pt}
\rhead{\footnotesize{\textit{\VERSION}}}

\makeatletter
\newcommand{\pushright}[1]{\ifmeasuring@#1\else\omit\hfill$\displaystyle#1$\fi\ignorespaces}
\newcommand{\pushleft}[1]{\ifmeasuring@#1\else\omit$\displaystyle#1$\hfill\fi\ignorespaces}
\makeatother

% Macros that make math operations more convenient to type.
\DeclareMathOperator*{\argmin}{arg\,min}
\DeclareMathOperator*{\argmax}{arg\,max}
\DeclareMathOperator*{\plim}{plim}

\DeclareMathOperator{\betarv}{\mathit{beta}}

\DeclareMathOperator{\1}{\mathbbm{1}}
\DeclareMathOperator{\E}{\mathbbm{E}}
\DeclareMathOperator{\V}{\mathbbm{V}}
\DeclareMathOperator{\tr}{tr}
\DeclareMathOperator{\diag}{diag}

\newcommand{\epb}{\boldsymbol{\varepsilon}}

\renewcommand{\Pr}{\mathbbm{P}}
\newcommand{\textif}{\text{if}}
\newcommand{\otherwise}{\text{otherwise}}

\newcommand{\Bf}{\mathfrak{B}}
\newcommand{\Bs}{\mathsf{B}}
\newcommand{\Fc}{\mathcal{F}}
\newcommand{\RR}{\mathbbm{R}}

% Local sum, average, and scaled average macros.
% - 's' stands for 'set index'
% - 'o' stands for 'ordered index'
\newcommand\ssum[3][\pm]{\sum_{#2 \in I_{#1}({#3})}}
\newcommand\savg[3][\pm]{\tfrac{1}{M_{#1}({#3})} \ssum[#1]{#2}{#3}}
\newcommand\sclt[3][\pm]{\tfrac{1}{\sqrt{M_{#1}({#3})}} \ssum[#1]{#2}{#3}}

\newcommand\osum[3][\pm]{\sum_{#2 = 1}^{M_{#1}({#3})}}
\newcommand\oavg[3][\pm]{\tfrac{1}{M_{#1}({#1})} \osum[#1]{#2}{#3}}
\newcommand\oclt[3][\pm]{\tfrac{1}{\sqrt{M_{#1}({#1})}} \osum[#1]{#2}{#3}}

% Abbreviations for monte carlo tables

\newcommand{\ccttri}{CCT (triangular)}
\newcommand{\cctuni}{CCT (uniform)}
\newcommand{\bootuni}{Resid.\ bootstrap}
\newcommand{\bootnaive}{Naive bootstrap}
\newcommand{\naiveuni}{Naive (uniform)}

% table spacing
\newcommand\T{\rule{0pt}{2.6ex}}       % Top strut
\newcommand\B{\rule[-1.2ex]{0pt}{0pt}} % Bottom strut

\newcommand{\Bft}{\tilde{\Bf}}
\newcommand{\Ct}{\tilde{C}}
\newcommand{\Gammat}{\tilde{\Gamma}}
\newcommand{\Psit}{\tilde{\Psi}}
\newcommand{\Vt}{\tilde{V}}


\title{Reply to the report by Referee 1 for ``Bootstrap Confidence Intervals for Sharp Regression Discontinuity Designs
  with the Uniform Kernel''}

\author{Ot\'avio Bartalotti \and Gray Calhoun \and Yang He\thanks{%
  All authors: Department of Economics, Iowa State University.
  260 Heady Hall, Ames, IA 50011.
  Bartalotti: \protect\url{bartalot@iastate.edu};
  Calhoun: \protect\url{gcalhoun@iastate.edu} and
  \protect\url{http://gray.clhn.org};
  He: \protect\url{yanghe@iastate.edu}.}}

\begin{document}
\maketitle

First, we thank the reviewer for providing detailed and helpful comments, which we feel have significantly improved the quality of the draft.  Below we provide replies to the reviewer’s specific comments detailing how we address each comment.

\section{General overview of changes}

For this version of the paper, we have changed the bootstrap method from the
residual bootstrap to the wild bootstrap. We have also updated the simulations
to include DGPs featuring heteroskedasticity. We believe that the residual
bootstrap used in the previous version is theoretically correct even under
heteroskedastcity as long as the conditional variance is assumed to be smooth in
a neighorbood around the cutoff point, since the residual bootstrap we initially
proposed only resampled from that neighborhood, but we agree that in practice
the wild bootstrap is likely to be more accurate and can accomodate forms of
conditional heteroskedasticity that the residual bootstrap can not.

We have also relaxed the ``uniform kernel'' requirement and allow other kernels
that are popular in practice. The title of the paper has been updated to reflect
that change.

The first referee expressed concern about the possible computational complexity of
using an iterated bootstrap to produce confidence intervals. This is much less
of an issue than it appears, because many of the matrix calculations can be
computed once and then reused in the bootstrap replications. (Any calculation
that does not involve $Y$ can be reused.) Each bootstrap replication only
requires a small number of matrix-vector multiplications which can be completed
very quickly. We have added a short section after Algorithm 2 that explains
this implementation detail, and we have also provided open source R code that
implements the procedure efficiently.

\section{Specific Comments}

\begin{enumerate}
 \item  \textit{``It is not clear what is special about the uniform kernel. If possible, the authors should consider a general kernel and remove Uniform kernel from the title.''}

 We added results that allow for more general kernels typically used in RDD. The title has been changed accordingly.

\item \textit{``...What are the potential benefits of bootstrap over the analytical calculations.''}

   \textbf{We need to decide what to answer here. In the current version of the paper (which I assume is what they read) we have a paragraph discussing it that needs to be updated with the kernel and wild bootstrap innovations. However it is not clear to me what else he wants or what we could write. Here is the full original paragraph. I will start editing it in the main file, but feel free to add your two cents if you have an idea. ``The methods presented in this paper are intended as a ``proof of concept'' that
demonstrates that the analytical correction proposed by CCT can also be
implemented through a new bootstrap procedure. We believe that this procedure is
promising because it can allow applied researchers to construct
accurate confidence intervals in settings where unusual forms of dependence
would make it difficult to derive CCT's analytical correction---the dependence
can be accommodates in principle by modifying the resampling algorithm to one
that matches the dependence in the dataset. The specific version of the bootstrap
presented in this paper, however, does not achieve that level of generality yet,
and two limitations should be noted in particular. First, the confidence
intervals produced by this bootstrap depend on two separate bandwidth parameters
that must be generated elsewhere. Since these bandwidths are generated using
analytical methods, the confidence intervals could easily be produced using the
same methods. Second, the bootstrap proposed in this paper imposes local
homoskedasticity and can perform poorly when the DGP exhibits significant
heteroskedasticity. Moreover, in this paper we focus on the baseline case of
sharp RD with a local linear model and uniform kernel. So the practical benefits
of using the bootstrap in this context are relatively limited. Relaxing these
limitations is the subject of ongoing research by the authors.''}

\item \textit{`` The most critical point: Assumption 1.4 allows for heteroskedasticity but I suspect the residual bootstrap is invalid in this setting...''}

 We thank the reviewer for focusing our attention to this point and suggesting the use of the Wild bootstrap. Following his recommendation we altered the algorithm to use a wild bootstrap approach and discuss its properties, hence allowing for conditional heteroskedasticity.

\item \textit{``Are the bootstrap residuals conditionally centered?...''}

 Yes, the bootstrap residuals are conditionally centered. Hopefully this is clear on the discussion about the wild bootstrap approach.

\item \textit{``I assume $X_{i}^{*}=X_{i}$, right? If so, use $X_{i}$''}

Thank you for this suggestion. We changed it on the latest draft.

\item \textit{``...Is it possible to propose a bootstrap approximation that does not require iterations...?''}

See discussion on item number 2 in the main comments section above.

\item \textit{``As mentioned earlier, I suspect Theorem 2 is incorrect unless conditional homoskedasticity is assumed... .''}

  Please see discussion for item 1 on the main comments section and item 3 in the specific comments above.

\item \textit{``The Monte Carlo should also include conditional heteroskedasticity versions of the used DGPs.''}

  Thank you for suggesting additional DGPs. We have performed new simulations using new DGPs with both mild and severe heteroskedasticity, please see the simulations section in the new draft.

\item \textit{``The proofs need to include more detailed steps. Be explicit about bootstrap parameters (e.g. $\tau^{*}$). Provide a detailed proof for the statement that follows after ``the bootstrap ensures that'' in page 19. What do you mean by the last formula of page 20? Also, it is not clear what equality in distribution in the first display of page 21 means. Where is $V(h,b)$ defined? I was unable to find it. The authors need to make sure that the convergence in distribution proof is valid under conditional heteroskedasticity.''}

Thank you very much for pointing out instances in the proofs that might require checking and clarifications. We are sorry this was confusing to read. In the new proofs we have added details to the statement ``the bootstrap ensures that'', previously on page 19. Since the proof for theorem 2 changed substantially with the adoption of the Wild Bootstrap scheme (which addresses the heteroskedasticity issue, as discussed before) we made substantial changes that hopefully will make the proof's steps clearer to the reader, addressing the remaining comments the reviewer pointed our attention to, including the definition of V(h,b). \textbf{After all is said and done remember to go over and check if all notation has been properly defined}.

\item \textit{``Some small typos: ``accommodates'' should be ``accommodated'' in line 4 of page 4; drop non-negligibly in page 7.''}
 
Thank you very much for pointing those out. They have been corrected.

\end{enumerate}


\end{document}