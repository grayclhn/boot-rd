\documentclass[12pt,fleqn]{article}
\input{VERSION}
\usepackage{
  amsfonts,
  amsmath,
  amssymb,
  amsthm,
  booktabs,
  fancyhdr,
  graphicx,
  newtxmath,
  newtxtext,
}

\frenchspacing
\usepackage[margin=1in]{geometry}

% Add version information to the paper's date and its page heading.
\date{\VERSION}
\pagestyle{fancy}
\renewcommand{\sectionmark}[1]{\markboth{}{\footnotesize{\thesection. #1}}}
\renewcommand{\subsectionmark}[1]{\markboth{}{\footnotesize{\thesubsection. #1}}}
\renewcommand{\headrulewidth}{0pt}
\renewcommand{\footrulewidth}{0pt}
\rhead{\footnotesize{\textit{\VERSION}}}


\title{Bootstrap confidence intervals for regression discontinuity estimators
  with the uniform kernel}
\author{Ot\'avio Bartalotti \and Gray Calhoun \and Yang He}

\begin{document}
\maketitle

\begin{abstract}
This paper proves some results.
\end{abstract}

\section{Introduction}
Regression Discontinuity (RD) designs have emerged in the last decade as an 
important and popular identification strategy to analyze the impact of policies
and interventions in several fields of the social sciences, including economics,
political science, public policy, sociology among others.

The strategy exploits arbitrary rules used to assign treatment to units, usually
relying on some type of ``score,'' the so called running variable. In its basic 
version, units with score above a certain treshold receive treatment while the 
ones below that cutoff are left untreated.

The identification of the treatment effect at the treshold is then based on 
comparing treated and untreated units at the cutoff. As a practical matter this 
involves comparing units within a bandwidth just above and just below the 
treshold. That comparison is most commonly done by implementing a local linear
estimator above and below the cutoff within bandwidths chosen to minimize the 
asymptotic mean squared error (AMSE) as proposed by Imbens and Kalyanaraman(2012)
and extensions by Calonico, Cattaneo and Titiunik (2014).

Recently, Calonico, Cattaneo and Titiunik (2014) brought attention to the fact
that the popular implementation of RD designs coupling optimal-AMSE bandwidths
and local polynomial estimation provides confidence intervals (CIs) with
incorrect coverage due to the presence of bias and propose an explicit analytical
bias correction procedure that re-centers and re-scale the CI by estimating a higher 
order approximation of the bias term and adjustment its lenght to account for 
the variation introduced by the bias estimation.

As pointed out by Calonico, Cattaneo and Farrell (2015), ``(...)valid inference
requires the delicate balancing act of selecting a bandwidth small enough
to remove smoothing bias, yet large enough to ensure adequate precision.''
The authors compare the strategies of undersmoothing, and explicit bias 
correction concluding that direct bias-correction with the corresponding 
rescalling of the CI's is a superior approach.

This paper contributes to the bias-corrected RD literature by proposing a simple
bootstrap procedure to correct the first order bias term and directly obtain 
valid CIs that provide the correct coverage. The algorithm proposed is easy to
implement and rely on the same conditions as the explicit bias correction
proposed by Calonico, Cattaneo and Titiunik (2014).

The novel bootstrap procedure is easy to implement and extends to RD designs 
using higher order polynomials, producing valid CIs which are robust to 
bandwidth choice and have similar properties to the analytical procedure in 
Calonico, Cattaneo and Titiunik (2014) while sidestepping the need to derive 
analytical formulas in each case of interest.

The paper is organized as follows. Section \ref{background} describes the basic
RD approach, its usual implementation, and the explicit analytical bias 
correction approach in the literature. Section \ref{boot} presents the proposed 
bootstrap bias corrected RD algorithm and discusses its properties. Simulation
evidence that the bootstrap procedure provides valid CIs and its relative 
performance to the analytical bias correction are presented in Section \ref{sim}.
Finally, Section \ref{conclusion} concludes.

\section{Background}\label{background}

\section{Bootstrap Bias Correction}\label{boot}

\section{Simulation Evidence}\label{sim}

\section{Conclusion}\label{conclusion}


\end{document}