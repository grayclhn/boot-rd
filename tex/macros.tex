\makeatletter
\newcommand{\pushright}[1]{\ifmeasuring@#1\else\omit\hfill$\displaystyle#1$\fi\ignorespaces}
\newcommand{\pushleft}[1]{\ifmeasuring@#1\else\omit$\displaystyle#1$\hfill\fi\ignorespaces}
\makeatother

% Macros that make math operations more convenient to type.
\DeclareMathOperator*{\argmin}{arg\,min}
\DeclareMathOperator*{\argmax}{arg\,max}
\DeclareMathOperator*{\plim}{plim}

\DeclareMathOperator{\1}{\mathbbm{1}}
\DeclareMathOperator{\E}{\mathbbm{E}}
\DeclareMathOperator{\V}{\mathbbm{V}}
\DeclareMathOperator{\tr}{tr}

\newcommand{\epb}{\boldsymbol{\varepsilon}}

\renewcommand{\Pr}{\operatorname{Pr}}
\newcommand{\textif}{\text{if}}
\newcommand{\otherwise}{\text{otherwise}}

\newcommand{\Bf}{\mathfrak{B}}
\newcommand{\Bs}{\mathsf{B}}
\newcommand{\Fc}{\mathcal{F}}
\newcommand{\RR}{\mathbbm{R}}

% Local sum, average, and scaled average macros.
% - 's' stands for 'set index'
% - 'o' stands for 'ordered index'
\newcommand\ssum[3][\pm]{\sum_{#2 \in I_{#3}^{#1}}}
\newcommand\savg[3][\pm]{\tfrac{1}{M_{#3}^{#1}} \ssum[#1]{#2}{#3}}
\newcommand\sclt[3][\pm]{\tfrac{1}{\sqrt{M_{#3}^{#1}}} \ssum[#1]{#2}{#3}}

\newcommand\osum[3][\pm]{\sum_{#2 = 1}^{M_{#3}^{#1}}}
\newcommand\oavg[3][\pm]{\tfrac{1}{M_{#3}^{#1}} \osum[#1]{#2}{#3}}
\newcommand\oclt[3][\pm]{\tfrac{1}{\sqrt{M_{#3}^{#1}}} \osum[#1]{#2}{#3}}

