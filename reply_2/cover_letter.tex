\documentclass[12pt,fleqn]{article}

% Add commit information from Git to the pdf. It is automatically
% generated with the R script 'tex/git and can be run from
% the command line in the project's top directory via:
%
% $ tex/git
%
% If VERSION.tex does not exist we can't add information from
% Git, so we'll use today's date as a fallback.

\IfFileExists{../VERSION.tex}{\input{../VERSION}}{%
\providecommand\VERSION{\today}}

\usepackage{
  amsfonts,
  amsmath,
  amssymb,
  amsthm,
  booktabs,
  fancyhdr,
  graphicx,
  newtxmath,
  newtxtext,
}

\frenchspacing
\usepackage[margin=1in]{geometry}

% Add version information to the paper's date and its page heading.
\date{\VERSION}
\pagestyle{fancy}
\renewcommand{\sectionmark}[1]{\markboth{}{\footnotesize{\thesection. #1}}}
\renewcommand{\subsectionmark}[1]{\markboth{}{\footnotesize{\thesubsection. #1}}}
\renewcommand{\headrulewidth}{0pt}
\renewcommand{\footrulewidth}{0pt}
\rhead{\footnotesize{\textit{\VERSION}}}

\makeatletter
\newcommand{\pushright}[1]{\ifmeasuring@#1\else\omit\hfill$\displaystyle#1$\fi\ignorespaces}
\newcommand{\pushleft}[1]{\ifmeasuring@#1\else\omit$\displaystyle#1$\hfill\fi\ignorespaces}
\makeatother

% Macros that make math operations more convenient to type.
\DeclareMathOperator*{\argmin}{arg\,min}
\DeclareMathOperator*{\argmax}{arg\,max}
\DeclareMathOperator*{\plim}{plim}

\DeclareMathOperator{\betarv}{\mathit{beta}}

\DeclareMathOperator{\1}{\mathbbm{1}}
\DeclareMathOperator{\E}{\mathbbm{E}}
\DeclareMathOperator{\V}{\mathbbm{V}}
\DeclareMathOperator{\tr}{tr}
\DeclareMathOperator{\diag}{diag}

\newcommand{\epb}{\boldsymbol{\varepsilon}}

\renewcommand{\Pr}{\mathbbm{P}}
\newcommand{\textif}{\text{if}}
\newcommand{\otherwise}{\text{otherwise}}

\newcommand{\Bf}{\mathfrak{B}}
\newcommand{\Bs}{\mathsf{B}}
\newcommand{\Fc}{\mathcal{F}}
\newcommand{\RR}{\mathbbm{R}}

% Local sum, average, and scaled average macros.
% - 's' stands for 'set index'
% - 'o' stands for 'ordered index'
\newcommand\ssum[3][\pm]{\sum_{#2 \in I_{#1}({#3})}}
\newcommand\savg[3][\pm]{\tfrac{1}{M_{#1}({#3})} \ssum[#1]{#2}{#3}}
\newcommand\sclt[3][\pm]{\tfrac{1}{\sqrt{M_{#1}({#3})}} \ssum[#1]{#2}{#3}}

\newcommand\osum[3][\pm]{\sum_{#2 = 1}^{M_{#1}({#3})}}
\newcommand\oavg[3][\pm]{\tfrac{1}{M_{#1}({#1})} \osum[#1]{#2}{#3}}
\newcommand\oclt[3][\pm]{\tfrac{1}{\sqrt{M_{#1}({#1})}} \osum[#1]{#2}{#3}}

% Abbreviations for monte carlo tables

\newcommand{\ccttri}{CCT (triangular)}
\newcommand{\cctuni}{CCT (uniform)}
\newcommand{\bootuni}{Resid.\ bootstrap}
\newcommand{\bootnaive}{Naive bootstrap}
\newcommand{\naiveuni}{Naive (uniform)}

% table spacing
\newcommand\T{\rule{0pt}{2.6ex}}       % Top strut
\newcommand\B{\rule[-1.2ex]{0pt}{0pt}} % Bottom strut

\newcommand{\Bft}{\tilde{\Bf}}
\newcommand{\Ct}{\tilde{C}}
\newcommand{\Gammat}{\tilde{\Gamma}}
\newcommand{\Psit}{\tilde{\Psi}}
\newcommand{\Vt}{\tilde{V}}


\begin{document}

\noindent%
Hi Matias,

\strut

Regarding referee 1's comments,
\begin{itemize}
\item the referee noticed that the bias and variance of the analytical
  correction is identical to the bootstrap correction. (Up to reported
  values --- they do differ in higher decimal places that were not
  reported.) This led us to examine our theoretical results more
  carefully and ultimately strengthen Theorem 1. Since both of the
  corrections are based on the same estimated 2nd order polynomial,
  the bootstrap and analytical corrections are not just asymptotically
  equivalent; they are equivalent up to simulation error.
\item We have also clarified the wording issues addressed in comment 2
  and explicitly addressed this coincidence.
\end{itemize}
Regarding referee 2's comments.
\begin{itemize}
\item The first comment pointed out that our bandwidth assumption was
  unnecessarily strong; we have relaxed it as suggested.
\item The referee's third comment alerted us to an error in the order
  of the variance of the bias-corrected estimator that was implicit in
  the statement of Theorem 2. We have fixed this error by explicitly
  dividing by the estimator's standard deviation in the statement of
  the theorem. We have also been more careful in how we apply these
  scale factors in the proof and have rewritten the last half-page
  of the proof to rescale several summations individually instead of
  \textit{en masse}, which fixes our initial errors.
\end{itemize}
We have also made the small notational changes and new definitions
the referees requested.

In addition, we discovered a small error in the simulation results ---
instead of reporting each estimator's bias, we reported the negative
bias. Since the values were close to zero, this does not have a
substantive effect on the results but we have, of course, fixed the
tables.

\noindent%
Best,

\noindent%
Ot\'avio Bartalotti, Gray Calhoun, and Yang He

\end{document}