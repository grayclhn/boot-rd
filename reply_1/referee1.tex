\documentclass[12pt,fleqn]{article}

% Add commit information from Git to the pdf. It is automatically
% generated with the R script 'tex/git and can be run from
% the command line in the project's top directory via:
%
% $ tex/git
%
% If VERSION.tex does not exist we can't add information from
% Git, so we'll use today's date as a fallback.

\IfFileExists{../VERSION.tex}{\input{../VERSION}}{%
\providecommand\VERSION{\today}}

\usepackage{
  amsfonts,
  amsmath,
  amssymb,
  amsthm,
  booktabs,
  fancyhdr,
  graphicx,
  newtxmath,
  newtxtext,
}

\frenchspacing
\usepackage[margin=1in]{geometry}

% Add version information to the paper's date and its page heading.
\date{\VERSION}
\pagestyle{fancy}
\renewcommand{\sectionmark}[1]{\markboth{}{\footnotesize{\thesection. #1}}}
\renewcommand{\subsectionmark}[1]{\markboth{}{\footnotesize{\thesubsection. #1}}}
\renewcommand{\headrulewidth}{0pt}
\renewcommand{\footrulewidth}{0pt}
\rhead{\footnotesize{\textit{\VERSION}}}

% Macros that make math operations more convenient to type.
\DeclareMathOperator*{\argmin}{arg\,min}
\DeclareMathOperator*{\argmax}{arg\,max}
\DeclareMathOperator*{\plim}{plim}

\DeclareMathOperator{\1}{\mathbbm{1}}
\DeclareMathOperator{\E}{\mathbbm{E}}
\DeclareMathOperator{\V}{\mathbbm{V}}

\renewcommand{\Pr}{\operatorname{Pr}}
\newcommand{\textif}{\text{if}}
\newcommand{\otherwise}{\text{otherwise}}

\newcommand{\Bf}{\mathfrak{B}}
\newcommand{\Bs}{\mathsf{B}}
\newcommand{\Fc}{\mathcal{F}}
\newcommand{\RR}{\mathbbm{R}}

% Local sum, average, and scaled average macros.
% - 's' stands for 'set index'
% - 'o' stands for 'ordered index'
\newcommand\ssum[3][\pm]{\sum_{#2 \in I_{#3}^{#1}}}
\newcommand\savg[3][\pm]{\tfrac{1}{M_{#3}^{#1}} \ssum[#1]{#2}{#3}}
\newcommand\sclt[3][\pm]{\tfrac{1}{\sqrt{M_{#3}^{#1}}} \ssum[#1]{#2}{#3}}

\newcommand\osum[3][\pm]{\sum_{#2 = 1}^{M_{#3}^{#1}}}
\newcommand\oavg[3][\pm]{\tfrac{1}{M_{#3}^{#1}} \osum[#1]{#2}{#3}}
\newcommand\oclt[3][\pm]{\tfrac{1}{\sqrt{M_{#3}^{#1}}} \osum[#1]{#2}{#3}}



\title{Reply to the report by Referee 1 for ``Bootstrap Confidence Intervals for Sharp Regression Discontinuity Designs
  with the Uniform Kernel''}

\author{Ot\'avio Bartalotti \and Gray Calhoun \and Yang He\thanks{%
  All authors: Department of Economics, Iowa State University.
  260 Heady Hall, Ames, IA 50011.
  Bartalotti: \protect\url{bartalot@iastate.edu};
  Calhoun: \protect\url{gcalhoun@iastate.edu} and
  \protect\url{http://gray.clhn.org};
  He: \protect\url{yanghe@iastate.edu}.}}

\begin{document}
\maketitle

First, we thank the reviewer for providing detailed and helpful comments, which
we feel have significantly improved the quality of the draft.  Below we provide
replies to the reviewer's specific comments detailing how we address each
comment.

\section{General overview of changes}

For this version of the paper, we have changed the bootstrap method from the
residual bootstrap to the wild bootstrap. We have also updated the simulations
to include DGPs featuring heteroskedasticity. We believe that the residual
bootstrap used in the previous version is theoretically correct even under
heteroskedastcity as long as the conditional variance is assumed to be smooth in
a neighborhood around the cutoff point, since the residual bootstrap we initially
proposed only resampled from that neighborhood, but we agree that in practice
the wild bootstrap is likely to be more accurate and can accommodate forms of
conditional heteroskedasticity that the residual bootstrap can not.

We have also relaxed the ``uniform kernel'' requirement and allow other kernels
that are popular in practice. The title of the paper has been updated to reflect
that change.

The first referee expressed concern about the possible computational complexity of
using an iterated bootstrap to produce confidence intervals. This is much less
of an issue than it appears, because many of the matrix calculations can be
computed once and then reused in the bootstrap replications. (Any calculation
that does not involve $Y$ can be reused.) Each bootstrap replication only
requires a small number of matrix-vector multiplications which can be completed
very quickly. We have added a short section after Algorithm 2 that explains
this implementation detail, and we have also provided open source R code that
implements the procedure efficiently.

Finally, the mathematical appendix defines more notation explicitly and
summarizes several convergence results from Calonico, Cattaneo, and Titiunik
(2014) that are referenced in the proofs and we have made some small additional
changes that the referees suggested.

\section{Specific Comments}

\begin{enumerate}
 \item  \textit{``It is not clear what is special about the uniform kernel. If possible, the authors should consider a general kernel and remove Uniform kernel from the title.''}

 We added results that allow for more general kernels typically used in RDD. The title has been changed accordingly.

\item \textit{``...What are the potential benefits of bootstrap over the analytical calculations.''}

  As now discussed in the text, we believe that this procedure is promising
  because of its ``automatic'' nature. It is easier to implement in cases of
  interest to applied researchers for which the analytical correction has not
  been developed; for example, analytical expressions for the standard errors
  with clustered observations have been only recently available despite heavy
  demand and interest from applied researchers, but would be relatively
  straightforward to calculate with a bootstrap.  We do not claim to achieve
  this level of generality in this paper but we expect that it will serve as a
  stepping stone for these developments.

\item \textit{`` The most critical point: Assumption 1.4 allows for heteroskedasticity but I suspect the residual bootstrap is invalid in this setting...''}

Thanks for raising this concern. See the discussion above.

\item \textit{``Are the bootstrap residuals conditionally centered?...''}

 Yes, the bootstrap residuals are conditionally centered. Hopefully this is clear on the discussion about the wild bootstrap approach.

\item \textit{``I assume $X_{i}^{*}=X_{i}$, right? If so, use $X_{i}$''}

Thank you for this suggestion. We changed it on the latest draft.

\item \textit{``...Is it possible to propose a bootstrap approximation that does not require iterations...?''}

See discussion on item number 2 in the main comments section above. The
iterations are less computationally demanding than they appear.

\item \textit{``As mentioned earlier, I suspect Theorem 2 is incorrect unless conditional homoskedasticity is assumed....''}

  Thanks again. See the discussion above.

\item \textit{``The Monte Carlo should also include conditional heteroskedasticity versions of the used DGPs.''}

  Thank you for this suggestion; we have added some additional simulations with conditional heteroskedasticity.

\item \textit{``The proofs need to include more detailed steps. Be explicit about bootstrap parameters (e.g. $\tau^{*}$). Provide a detailed proof for the statement that follows after ``the bootstrap ensures that'' in page 19. What do you mean by the last formula of page 20? Also, it is not clear what equality in distribution in the first display of page 21 means. Where is $V(h,b)$ defined? I was unable to find it. The authors need to make sure that the convergence in distribution proof is valid under conditional heteroskedasticity.''}

Thank you for calling our attention to these issues.
 In the new proofs we have added details to the statement ``the bootstrap ensures that,'' previously on page 19.
 Since the proof for theorem 2 changed substantially with the adoption of the Wild Bootstrap scheme
 (which addresses the heteroskedasticity issue, as discussed before) we made substantial changes that hopefully
 will make the proof's steps clearer to the reader, addressing the remaining comments.

\item \textit{``Some small typos: ``accommodates'' should be ``accommodated'' in line 4 of page 4; drop non-negligibly in page 7.''}
 
Thanks. They have been fixed.

\end{enumerate}


\end{document}